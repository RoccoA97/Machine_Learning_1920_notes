\providecommand{\main}{../../main}
\providecommand{\figpath}[1]{\main/../lessons/#1}
\documentclass[../../main/main.tex]{subfiles}



\begin{document}

In \( \mathcal{L} \) there are 4 discrete symmetries/transformations:
\begin{itemize}
    \item Trivial: \( \mathbbm{1} \)

    \item Space inversion \( I_S \) such as: \( I_S x^{\mu} = g_{\mu \nu} x^{\mu} = x_{\mu} = (x_0, -\va{x} ) \)

    \item Time inversion \( I_t \): \( I_t x^{\mu } = - g_{\mu \nu} x^{\nu} = (-x_0, \va{x} ) \)

    \item Space-time inversion \( I_{st} \): \( I_{st} = I_s I_t = I_t I_s  \qquad I_{st} x^{\mu} = -x^{\mu} \)
\end{itemize}

We can define \( \mathcal{I} = \{ \mathbbm{1}, I_s, I_t, I_{st} \} \), which is a non-invariant subgroup. It's abelian though, and not connected. So, the \( \mathcal{L} \) components can be summarized in a table such as:

\begin{table}[h]
    \centering
    \begin{tabular}{|c|c|c|c|}
        \hline
        \( \mathcal{L} \) components    &   \( \det(\Lambda )  \)   &   \( \Lambda^{0}_{\ 0}  \)    &   Discrete transformation \\
        \hline
        \( \mathcal{L}^{\uparrow}_{+} \)    &   1  &   \( \ge 1 \) &   \( \mathbbm{1} \)   \\
        \( \mathcal{L}^{\uparrow}_{-} \)    &   -1  &   \( \ge 1 \) &   \( I_s = g \)   \\
        \( \mathcal{L}^{\downarrow}_{+} \)    &   1  &   \( \le -1 \) &   \( I_t = -g \)   \\
        \( \mathcal{L}^{\downarrow}_{-} \)    &   -1  &   \( \le -1 \) &   \( I_{st} = -\mathbbm{1} \)  \\
        \hline
    \end{tabular}
    \caption{Components of \( \mathcal{L} \) }
    \label{}
\end{table}

\( \mathcal{L}^{\uparrow}_{+} \) group contains two types of continuous transformations:
\begin{itemize}
    \item \textbf{Space rotations}, which depend on 3 d.o.f., so on the three angles of rotation:
        \[
            R =
            \left(
            \begin{array}{c|c}
            1   & \begin{matrix} 0 & 0 & 0 \end{matrix} \\
            \hline
            \begin{matrix} 0 \\ 0 \\ 0 \end{matrix}   & R_{ij}
            \end{array}
            \right)
        \]

    \item \textbf{Pure Lorentz transformations (boosts)}. For example:
        \[
            \begin{rcases}
                t' = \gamma \left(t - \frac{v}{c^2}x \right) \\
                x' = \gamma (x-vt)  \\
                y' = y  \\
                z' = z
            \end{rcases}
            \iff
            {x'}^{\mu} \equiv {L_1}^{\mu}_{\ \nu} x^{\nu}
            \qquad
            L_1 =
            \begin{pmatrix}
            \gamma   & -\beta \gamma  & 0  & 0 \\
            -\beta \gamma   & \gamma  & 0  & 0 \\
            0   & 0  & 1  & 0 \\
            0   & 0  & 0  & 1
            \end{pmatrix}
        \]
        where \( \beta = \frac{v}{c} \) and \( \gamma = \frac{1}{\sqrt{1 - \beta^2} }  \). We can also consider \( \gamma = \cosh{\psi}  \) in order to have a rotation of a complex angle \( \theta = i \psi \).

    \item \textbf{Generic boost}:
        \[
            L^{\mu }_{\ \nu } =
            \left(
            \begin{array}{c|c}
                \gamma &    \beta_j \gamma \\
                \hline
                -\beta _i \gamma &  \delta ^{i}_{\ j} - \frac{\beta ^i \beta _j}{\beta ^2}(\gamma -1)
            \end{array}
            \right)
        \]
        So \( \mathcal{L}^{\uparrow }_{+} \) depends on 6 d.o.f.. Moreover, \( \mathcal{L}^{\uparrow }_{+} \ni \Lambda \), we have: \( \Lambda = LR \).
\end{itemize}

We are going to enstablish a connection between \( SO(1,3) \) and \( SL(2,\C) \). So we want to find a homomorphism \( SO(1,3) \longrightarrow SL(2,\C) \):

\[
    x^\mu = (x^0,x^1,x^2,x^3)
    \longrightarrow
    X = \sigma _{\mu} x^{\mu} =
    \begin{pmatrix}
    x^0+x^3   & x^1 - ix^2 \\
    x^1 + ix^2   & x^0 - x^3
    \end{pmatrix}
    \qquad
    \begin{array}{c}
        \sigma _\mu \equiv (\mathbbm{1},\va{\sigma} )   \\
        \sigma ^\mu \equiv (\mathbbm{1},-\va{\sigma } )
    \end{array}
\]

We get the following results from trace computing:

\begin{align*}
    \tr(\sigma ^\mu \sigma_\nu) &= 2 g_{\mu \nu} \\
    \tr(\sigma _i \sigma _j)  &= 2 \delta _{ij}
\end{align*}

Take now \( A \in SL(2,\C) \):

\[
    A =
    \begin{pmatrix}
    \alpha   & \beta \\
    \gamma    & \delta
    \end{pmatrix}
    \qquad
    \begin{array}{c}
        \alpha, \beta, \gamma, \delta \in \C    \\
        \det(\Lambda) = 1 \Longrightarrow \alpha \delta - \beta \gamma = 1
    \end{array}
\]

If we transform the matrix \( X \) trough the matrix \( A \), we get:

\begin{equation}
    X' = A X A ^\dag
    \Longrightarrow
    \Lambda^{\mu}_{\ \nu} = \frac{1}{2} \tr(\sigma ^\mu A \sigma _\nu A ^\dag )
    \label{eq:}
\end{equation}

What we have found is that through the homomorphism we have a 2-to-1 correspondence between \( SL(2,\C) \) and \( \mathcal{L}^{\uparrow}_{+} \), in fact \( A \) and \( -A \) are associated to \( \Lambda \).

Concerning the Lie algebra of the group, there are 6 generators:
\begin{itemize}
    \item \textbf{Rotations}:
        \begin{equation}
            J_k \equiv i \left. \pdv{R_k}{\varphi} \right|_{\varphi = 0}
            \qquad
            R = \overbrace{\mathbbm{1} - i \delta \varphi \va{J} \cdot \va{n} }^{\text{Infinitesimal rotation}}
            = \overbrace{e^{-i \varphi \va{J} \cdot \va{n}} }^{\text{Finite rotation}}
            \label{eq:}
        \end{equation}

    \item \textbf{Boosts}:
    \begin{equation}
        K_l \equiv i \left. \pdv{L_l}{\psi} \right|_{\psi = 0}
        \qquad
        L = \overbrace{\mathbbm{1} - i \delta \psi \va{k} \cdot \va{\nu} }^{\text{Infinitesimal boost}}
        = \overbrace{e^{-i \varphi \va{k} \cdot \va{\nu}} }^{\text{Finite boost}}
        \label{eq:}
    \end{equation}
\end{itemize}

The commutators are:

\begin{subequations}
    \begin{align}
        [J_i,J_j]   &=  i \varepsilon _{ijk} J_k    \\
        [K_i,K_j]   &=  (-1) \varepsilon _{ijk} J_k \\
        [J_i,K_j]   &=  i \varepsilon _{ijk} K_k
    \end{align}
    \label{}
\end{subequations}

We can construct the antisymmetric tensor \( M_{\mu \nu} \) with the previous generators:

\begin{equation}
    M_{\mu \nu } =
    \begin{pmatrix}
    0   & K_1  & K_2  & K_3 \\
    -K_1   & 0  & J_3  & -J_2 \\
    -K_2   & -J_3  & 0  & J_1 \\
    -K_3   & J_2   & -J_1   & 0
    \end{pmatrix}
    \label{eq:}
\end{equation}

where \( J_i = \frac{1}{2} \varepsilon _{ijk} M_{jk} \) and \( K_i = M_{0i} \). Concerning the Lie algebra of the group:

\begin{equation}
    [M_{\mu \nu }, M_{\lambda \rho }] =
    i (
    g_{\lambda \mu } M_{\rho \nu } + g_{\rho \nu } M_{\lambda \mu}
    - g_{\lambda \nu } M_{\rho \mu } - g_{\rho \mu} M_{\lambda \nu }
    )
    \label{eq:}
\end{equation}

We can write in general \( \Lambda \) thorugh the antisymmetric tensor (whose elements are real) \( \omega _{\mu \nu } \):

\[
    \Lambda = e^{-\frac{i}{2} \omega _{\mu \nu } M^{\mu \nu }}
\]

An infinitesimal transformation can be written as:
\[
    \Lambda^{\mu}_{\ \nu } = \delta^{\mu }_{\ \nu} + \omega ^{\mu }_{\ \nu }
\]

\[
    \Lambda ^T g \Lambda = g \ :
    \qquad
    (\delta^{\mu }_{\ \rho} + \omega ^{\mu }_{\ \rho })
    g_{\mu \nu}
    (\delta^{\nu }_{\ \sigma} + \omega ^{\nu }_{\ \sigma })
    =
    g_{\rho \sigma}
    \Longrightarrow
    g_{\mu \sigma } + \omega _{\rho \sigma } + \omega _{\sigma \rho } = g_{\rho \sigma}
\]

Considering that the tensor \( \omega_{\mu \nu} \) is antisymmetric, it holds:

\[
    \omega_{\mu \nu} = \frac{1}{2} (\omega_{\mu \nu} - \omega_{\nu \mu})
    =
    \frac{1}{2} \omega _{\alpha \beta} (g^{\alpha}_{\ \mu } g^{\beta}_{\ \nu} - g^{\alpha}_{\ \nu } g^{\beta}_{\ \mu})
    \equiv
    - \frac{1}{2} i \omega _{\alpha \beta} (M^{\alpha \beta})_{\mu \nu}
\]

\begin{equation}
    (M^{\alpha \beta})_{\mu \nu} =
    i (g^{\alpha}_{\ \mu } g^{\beta}_{\ \nu} - g^{\alpha}_{\ \nu } g^{\beta}_{\ \mu})
    \label{eq:}
\end{equation}

So we can write again the tensor \( \Lambda \):

\[
    \Lambda^{\mu}_{\ \nu} = \delta^{\mu}_{\ \nu} - \frac{i}{2} \omega_{\alpha \beta } (M^{\alpha \beta})_{\mu \nu}
    \Longrightarrow
    \Lambda = \mathbbm{1} - \frac{i}{2} \omega_{\alpha \beta } M^{\alpha \beta}
    \longrightarrow
    e^{-\frac{i}{2} \omega_{\alpha \beta } M^{\alpha \beta}}
\]

Now we focu our attention on irreducible representations, in particular, we consider Casimir operators:

\begin{subequations}
    \begin{align}
        C_1 &= \frac{1}{2} M^{\mu \nu } M_{\mu \nu } = \va{J} ^2 - \va{K} ^2    \\
        C_2 &= \frac{1}{2} \varepsilon ^{\mu \nu \rho \sigma} M_{\mu \nu } M_{\rho \sigma } = - \va{J} \cdot \va{K}
    \end{align}
    \label{}
\end{subequations}

Alternatively:

\begin{equation*}
    \begin{rcases}
        M_i \equiv \frac{1}{2} (J_i + iK_i) \equiv J^+  \\
        N_i \equiv \frac{1}{2} (J_i - iK_i) \equiv J^-
    \end{rcases}
    \Longrightarrow
    \begin{aligned}
        [M_i,N_j] &= 0   \\
        [M_i,M_j] &= i \varepsilon _{ijk} M_k    \\
        [N_i,N_j] &= i \varepsilon _{ijk} N_k    \\
        [M^2,M_i] &= 0   \\
        [N^2,N_i] &= 0
    \end{aligned}
\end{equation*}

Concerning the algebra:

\begin{align*}
    so(1,3) &\simeq su(2) \oplus su(2)   \\
    SO(1,3) &\simeq SU(2) \otimes SU(2)
\end{align*}

In terms of representations and noticing that \( SO(3) \sim SU(2) \longrightarrow D^{(j)} \):
\[
    D^{(M,N)} \equiv D^{(M,0)} \oplus D^{(0,N)}
\]
So, starting from \( D^{(\frac{1}{2},\frac{1}{2})} \equiv D^{(\frac{1}{2},0)} \oplus D^{(0,\frac{1}{2})}\):

\[
    D^{(j_1,j_2)} \otimes D^{(j_1',j_2')} =
    D^{(j_1 + j_1',j_2 + j_2')} \oplus
    D^{(j_1 + j_1',j_2 + j_2' - 1)} \oplus
    \dots
    D^{(\abs{j_1 - j_1'},\abs{j_2 - j_2'})}
\]

Let's consider \( D^{(M,N)} \):

\begin{table}[h]
    \centering
    \begin{tabular}{|c|c|c|}
        \hline
        \( (M,N) \) & dim   &   Field (particle)    \\
        \hline
        \( (0,0) \) & 1     &   Scalar      \\
        \( (\frac{1}{2},0) \) & 2     &   Left handed spinor      \\
        \( (0,\frac{1}{2}) \) & 2     &   Right handed spinor      \\
        \( (\frac{1}{2},\frac{1}{2}) \) & 4     &   Vector      \\
        \( (1,0) \) & 3     &   Self dual (2-form) field      \\
        \( (0,1) \) & 3     &   Anti-self dual (2-form) field      \\
        \( (1,1) \) & 9     &   Traceless symmetric tensor field    \\
        \hline
    \end{tabular}
    \caption{Examples}
    \label{}
\end{table}



\section{Poincaré transformations \( \mathcal{P} \ni (a,\Lambda) \)}
The inhomogeneous Lorentz transformations, or Poincaré transformations, connect the space-time coordinates of any two frames of reference whose relative velocity is constant. In general, a Poincaré transformation can be written as a generalization of a Lorentz transformation in the form:
\begin{equation}
    {x'}^\mu = \Lambda^{\mu}_{\ \nu} x^\nu + a^\mu
    \label{eq:POINC}
\end{equation}
where \( a^\mu  \) stand for the components of a vector in \( \R^4 \). We will denote this kind of transformation by \( (a,\Lambda) \). They form a group.

In general, an infinitesimal translation is given by:
\begin{equation}
    (\delta a, \mathbbm{1}) = \mathbbm{1} - i \delta a_\mu P^\mu
    \label{eq:INF_TRAN_POINC}
\end{equation}
with the four operators \( P^\mu \) the infinitesimal generators of the translations. A finite translation instead is given by the exponentiation:
\begin{equation}
    (a,\mathbbm{1}) = e^{-i a_\mu P^\mu}
    \label{eq:FIN_TRAN_POINC}
\end{equation}

In order to get the commutation relations of the infinitesimal generators, it is convenient to use a \( 5\times 5 \) formalism. So we write \( (a,\Lambda ) \) as a \( 5 \times 5 \) matrix:
\[
    (a, \Lambda ) =
    \left(
    \begin{array}{c|c}
        \Lambda &   a   \\
        \hline
        0   &   1
    \end{array}
    \right)
    \qquad
    y^\mu = \begin{pmatrix}
    x^\mu \\
    1
    \end{pmatrix}
\]
Therefore, we can rewrite the transformation:
\begin{equation}
    \begin{pmatrix}
    x' \\
    1
    \end{pmatrix}
    =
    \begin{pmatrix}
    \Lambda    & a  \\
    0   & 1
    \end{pmatrix}
    \begin{pmatrix}
    x \\
    1
    \end{pmatrix}
    =
    \begin{pmatrix}
    \Lambda x +a \\
    1
    \end{pmatrix}
    \label{eq:}
\end{equation}

Expressing in a similar way the infinitesimal transformation in eq. \ref{eq:INF_TRAN_POINC}, one can obtain explicitly the four generators:
\begin{equation}
    \begin{gathered}
        P_0 =
        \begin{pmatrix}
        0   & 0  & 0  & 0 & i \\
        0   & 0  & 0  & 0 & 0 \\
        0   & 0  & 0  & 0 & 0 \\
        0   & 0  & 0  & 0 & 0 \\
        0   & 0  & 0  & 0 & 0
        \end{pmatrix}
        \qquad
        P_1 =
        \begin{pmatrix}
        0   & 0  & 0  & 0 & 0 \\
        0   & 0  & 0  & 0 & i \\
        0   & 0  & 0  & 0 & 0 \\
        0   & 0  & 0  & 0 & 0 \\
        0   & 0  & 0  & 0 & 0
        \end{pmatrix}   \\
        P_2 =
        \begin{pmatrix}
        0   & 0  & 0  & 0 & 0 \\
        0   & 0  & 0  & 0 & 0 \\
        0   & 0  & 0  & 0 & i \\
        0   & 0  & 0  & 0 & 0 \\
        0   & 0  & 0  & 0 & 0
        \end{pmatrix}
        \qquad
        P_3 =
        \begin{pmatrix}
        0   & 0  & 0  & 0 & 0 \\
        0   & 0  & 0  & 0 & 0 \\
        0   & 0  & 0  & 0 & 0 \\
        0   & 0  & 0  & 0 & i \\
        0   & 0  & 0  & 0 & 0
        \end{pmatrix}
    \end{gathered}
    \label{eq:}
\end{equation}

In the same representation, the generators \( J_i \) and \( K_i \) are replaced by \( 5 \times 5\) matrices, that can be obtained by the combination of two block: the first one is \( (J_i)_{4 \times 4} \) or \( (K_i)_{4 \times 4} \), the second one is \( 1 \).

Concerning the Lie algebra, the following commutation relations hold:
\begin{equation}
    \begin{gathered}
        \begin{aligned}
        [P_\mu,P_\nu] &= 0  \\
        [M_{\lambda \rho}, M_{\mu \nu}] &= -i (g_{\lambda \mu}M_{\rho \nu} + g_{\rho \nu}M_{\lambda \mu} - g_{\lambda \nu}M_{\rho \mu} - g_{\rho \mu}M_{\lambda \nu})    \\
        [M_{\mu \nu},P_{\rho}] &= - i (g_{\mu \nu} P_{\rho} - g_{\mu \rho} P_{\mu})
        \end{aligned}
    \end{gathered}
    \label{eq:}
\end{equation}

The subgroup \( \mathcal{P}^{\uparrow}_{+} \) has \( 10 = 6 + 4 \) generators.

Now, we introduce the Casimir operators \( C_1 \) and \( C_2 \), and \( W_\mu \), namely \textbf{Pauli-Lubarsky tensor}:
\begin{align}
    C_1 &= P^2 = P^\mu P_\mu   \\
    C_2 &= W^2 = W_\mu W^\mu   \\
    W_\mu &= \frac{1}{2} \varepsilon _{\mu \nu \sigma \tau} M^{\nu \sigma} P^{\tau}
\end{align}
It follows immediately that:
\begin{itemize}
    \item \( W_\mu P^\mu = 0 = \varepsilon _{\mu \nu \rho \tau} M^{\nu \sigma} P^{\tau } P^{\mu} \)
    \item \( [P_\mu, W_\mu] = 0 \)\\
        \( [M_{\mu \nu}, W_{\sigma}] = - i (g_{\nu \sigma}W_\mu - g_{\mu \sigma}W_\nu)\)\\
        \( [W_\mu,W_\nu] = i \varepsilon _{\mu \nu \sigma \tau} W^{\sigma}P^\tau \)
\end{itemize}



\end{document}
