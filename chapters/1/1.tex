\providecommand{\main}{../../main}
\providecommand{\figpath}[1]{\main/../chapters/#1}
\documentclass[../../main/main.tex]{subfiles}



\begin{document}

\chapter{Machine Learning framework}

We begin with the description of a formal model in order to capture what could be the learning tasks. The fundamental points are:

\begin{itemize}
    \item \textbf{The learner's input}:
    \begin{itemize}
        \item A domain set \( \mathcal{X} \), whose points are the istances we want to label.
        \item A label set \( Y \).
        \item The training dataset \( S = \mathcal{X} \times \mathcal{Y} \). It is a finite sequence of label domani points.
    \end{itemize}
    
    \item \textbf{The learner's output}:
    \begin{itemize}
        \item A prediction rule \( h: \mathcal{X} \to \mathcal{Y} \) (also called predictor or hyothesis or classifier). It is used to predict the label of new domain points. Therefore \( A(S) \) is the hypothesis, where \( A \) represents the algorithm.

        \item \textbf{Simple data-generation model}:\\
            Assume that the training data are generated by a probability distribution \( \mathbb{D} \) (over \( \mathcal{X} \)). Moreover, suppose that the learner doesn't know anything about the distribution and that there exists some 
    \end{itemize} 
\end{itemize}




\end{document}