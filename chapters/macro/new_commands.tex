% Derivatives
\renewcommand{\d}[0]{\mathrm{d}}
\newcommand{\dev}[2]{\displaystyle \frac{\d #1}{\d #2}}
\newcommand{\pdev}[2]{\displaystyle \frac{\partial #1}{\partial #2}}
\newcommand{\ndev}[3]{\displaystyle \frac{\d^{#3} #1}{\d #2^{#3} } }
\newcommand{\npdev}[3]{\displaystyle \frac{\partial^{#3} #1}{\partial #2^{#3} } }

%% Math operators
\DeclareMathOperator{\arctanh}{arctanh}
\DeclareMathOperator{\sign}{sign}

%% Norms
\newcommand{\absvec}[1]{| \vec{#1} |}
\newcommand{\normvec}[1]{|\!| #1 |\!|}

%% User defined commands
\newcommand{\mathcolorbox}[2]{\colorbox{#1}{$\displaystyle #2$}}
\newcommand{\hlfancy}[2]{\sethlcolor{#1}\hl{#2}}

%% User defined math commands
\newcommand{\R}{\mathbb{R}}
\newcommand{\C}{\mathbb{C}}
\newcommand{\vmed}[1]{\left \langle #1 \right \rangle}
\newcommand{\vmedvec}[1]{\langle #1 \rangle}
\newcommand{\id}{\mathbbm{1}}
\renewcommand{\exp}[1]{\operatorname{e}^{#1}}
\renewcommand{\H}[0]{\operatorname{H}}
\renewcommand{\Re}[1]{\operatorname{\mathbb{R}e}\left[ #1 \right]}
\renewcommand{\Im}[1]{\operatorname{\mathbb{I}m}\left[ #1 \right]}

%% Theorem environment
\theoremstyle{plain}
\newtheorem{theorem}{Theorem}[section]
\newtheorem{corollary}{Corollary}[theorem]
\newtheorem{lemma}[theorem]{Lemma}
\newtheorem{proposition}{Proposition}[section]
\newtheorem{axiom}{Axiom}
\newtheorem{postulate}{Postulate}

\theoremstyle{definition}
\newtheorem{definition}{Definition}[section]

\theoremstyle{remark}
\newtheorem{remark}{Remark}
\newtheorem{example}{Example}
\newtheorem{exercise}{Exercise}

%% TikZ
\newcommand*\circled[1]{\tikz[baseline=(char.base)]{\node[shape=circle,draw,inner sep=2pt] (char){#1};}}

%Evidenziare testo
\newcommand\mybox[1]{%
  \fbox{\begin{minipage}{0.9\textwidth}#1\end{minipage}}}

%Spiegazioni/verifiche
\newenvironment{greenbox}{\begin{mdframed}[hidealllines=true,backgroundcolor=green!20,innerleftmargin=3pt,innerrightmargin=3pt]}{\end{mdframed}}
\newenvironment{bluebox}{\begin{mdframed}[hidealllines=true,backgroundcolor=blue!10,innerleftmargin=3pt,innerrightmargin=3pt]}{\end{mdframed}}
\newenvironment{yellowbox}{\begin{mdframed}[hidealllines=true,backgroundcolor=yellow!20,innerleftmargin=3pt,innerrightmargin=3pt]}{\end{mdframed}}
\newenvironment{redbox}{\begin{mdframed}[hidealllines=true,backgroundcolor=red!20,innerleftmargin=3pt,innerrightmargin=3pt]}{\end{mdframed}}
\newenvironment{orangebox}{\begin{mdframed}[hidealllines=true,backgroundcolor=orange!20,innerleftmargin=3pt,innerrightmargin=3pt]}{\end{mdframed}}
