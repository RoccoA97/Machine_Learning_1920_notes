\providecommand{\main}{../../main}
\providecommand{\figpath}[1]{\main/../chapters/#1}
\documentclass[../../main/main.tex]{subfiles}



\begin{document}

\chapter*{Introduction}




\section*{Programme of the course}

Arguments treated:

\begin{enumerate}
	\item \textbf{Motivation}: components of the learning problem and applications of Machine Learning. Supervised and unsupervised learning.
	\item \textbf{Introduction}: the supervised learning problem, data, classes of models, losses.
	\item \textbf{Probabilistic models and assumptions on the data}: the regression function. Regression and classification.
	\item \textbf{When is a model good?}: model complexity, bias variance tradeoff/generalization (VC dimension, generalization error).
	\item \textbf{Models for regression}: linear regression (scalar and multivariate), subset selection, linear-in-the-parameters models, regularization.
	\item \textbf{Simple models for classification}: logistic regression, perceptron, naïve bayes classifier.
	\item \textbf{Kernel methods}: Support Vector Machines.
	\item \textbf{Neural Networks}.
	\item \textbf{Deep Learning}: Convolutional Neural Networks.
	\item \textbf{Validation and model selection}: generalization error, bias-variance tradeoff, cross validation. Model complexity determination.
	\item \textbf{Unsupervised learning}: cluster analysis, K-means clustering, EM estimation.
	\item \textbf{Dimensionality reduction}: Principal Component Analysis (PCA).
\end{enumerate}

\end{document}
